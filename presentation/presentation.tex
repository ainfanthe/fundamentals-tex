\documentclass{article}

\usepackage[utf8]{inputenc} % Indica cuál es la codificación de este archivo
\usepackage[spanish]{babel} % Indica el idioma en que está escrito el documento
\usepackage{graphicx}

% Redefino algunos nombres para reemplazar los de Babel Spanish
\addto\captionsspanish{
  \renewcommand\figurename{Gráfico}
  \renewcommand\listfigurename{Lista de gráficos}
  \renewcommand\tablename{Tabla}
  \renewcommand\listtablename{Lista de tablas}
}

\begin{document}

% Ingreso contenido de la caratula estándar de Latex
\title{Un documento de ejemplo} % Título del documento
\date{\today} % Fecha del documento. El comando \today inserta la fecha actual
\author{Andrés Hernández\thanks{Universidad Nacional de Colombia}} % Autor del documento

\maketitle % Genera la caratula

\newpage % Inicia una nueva página

\tableofcontents % Genera la tabla de contenido del documento

\listoffigures % Genera la lista de figuras del documento

\listoftables % Genera la lista de tablas del documento

\newpage % Inicia una nueva página

\section*{Introducción}
\addcontentsline{toc}{section}{Introducción}

Con este ejemplo se aprende sobre LaTeX:

\begin{itemize}
  \item Tabla de contenidos (índices)
  \item Carátulas
  \item Gráficos y entorno de figuras
\end{itemize}

\section{Lorem ipsum dolor sit amet, consectetur adipiscing elit. Suspendisse ac.}

\subsection[Fusce in]{Lorem ipsum dolor sit amet, consectetur adipiscing elit. Fusce in.}

\section{Tablas}

\subsection{Entorno tabular}

\subsubsection{Tabla simple sin divisiones}

\begin{tabular}{c c}
1 & 2 \\
3 & 4 \\
\end{tabular}

\subsubsection{Tabla con divisiones y fila de encabezado}

\begin{tabular}{| l | c | r |}
\hline
Alpha & Beta & Gama \\
\hline \hline
1 & 2 & 3 \\
\hline
4 & 5 & 6 \\
\hline
\end{tabular}

\subsubsection{Tabla con párrafo}

\begin{tabular}{| l | p{2.5cm} |}
\hline
Item & Descripción \\
\hline \hline
1 & Descripción de la primer fila \\
\hline
4 & Descripción de la segunda fila \\
\hline
\end{tabular}

\subsection{Entorno table}

\begin{table}[h]
  \centering
  \begin{tabular}{| l | p{2.5cm} |}
  \hline
  Item & Descripción \\
  \hline \hline
  1 & Descripción de la primer fila \\
  \hline
  4 & Descripción de la segunda fila \\
  \hline
  \end{tabular}
  \caption{Tabla de ejemplo}
  \label{tab:tabla}
\end{table}

\end{document}
